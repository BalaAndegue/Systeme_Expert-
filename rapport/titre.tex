% ============================================================
%  PAGE DE TITRE
% ============================================================
\begin{titlepage}
\centering

\vspace*{1cm}

% Logo / Emblème
\begin{tikzpicture}
  \draw[AgriGreen, line width=3pt] (0,0) circle (2cm);
  \node[AgriGreen, font=\Huge] at (0,0.3) {\faLeaf};
  \node[AgriBlue, font=\large\bfseries] at (0,-0.7) {AgriExpert};
\end{tikzpicture}

\vspace{1cm}

{\Huge\bfseries\color{AgriBlue}
Système Expert Multi-Agents\\[0.4cm]
pour l'Agriculture au Cameroun
}

\vspace{0.5cm}
\rule{\linewidth}{2pt}
\vspace{0.3cm}

{\Large\color{AgriGray}
Intégration IoT, Machine Learning et Intelligence Artificielle\\
pour le Suivi Intelligent des Cultures et Parcelles Agricoles
}

\vspace{0.5cm}
\rule{\linewidth}{1pt}

\vspace{1.5cm}

\begin{tcolorbox}[colback=LightGreen,colframe=AgriGreen,width=0.85\textwidth]
\centering
\textbf{Rapport Technique Détaillé}\\[0.3cm]
\begin{tabular}{ll}
\textbf{Domaine :} & Agriculture de précision / IA embarquée \\
\textbf{Technologie :} & Python, Flask, LLM (Gemini 2.0 Flash) \\
\textbf{Couverture :} & 10 régions du Cameroun \\
\textbf{Agents :} & 6 agents spécialisés \\
\textbf{Interface :} & CLI + API REST \\
\end{tabular}
\end{tcolorbox}

\vfill

\begin{tabular}{rl}
\textbf{Auteur :} & Équipe Projet AgriExpert Cameroun \\
\textbf{Date :} & Février 2026 \\
\textbf{Version :} & 2.0 (Optimisée) \\
\textbf{Statut :} & \textcolor{AgriGreen}{\textbf{Opérationnel}} \\
\end{tabular}

\vspace{1cm}
{\small\color{AgriGray}
Université / Institution — Département Informatique \& Génie Agricole
}

\end{titlepage}
