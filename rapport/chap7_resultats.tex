% ============================================================
%  CHAPITRE 7 — RÉSULTATS ET ÉVALUATION
% ============================================================
\chapter{Résultats et Évaluation du Système}

\section{Scénarios de Test Réels}

\subsection{Scénario 1 : Diagnostic Phytosanitaire}

\begin{warningbox}[Requête utilisateur]
\textit{« Mes feuilles de cacao ont des taches brunes et tombent. Que faire ? »}\\
Région : Centre | Parcelle : 2 ha
\end{warningbox}

\begin{successbox}[Réponse du système (HealthAgent + CropAgent)]
\textbf{Diagnostic :} Pourriture brune (\textit{Phytophthora palmivora}) — Gravité : ÉLEVÉE\\[0.3cm]
\textbf{Symptômes confirmés :} Taches brunes irrégulières sur feuilles et cabosses, chute précoce\\[0.3cm]
\textbf{Traitement immédiat :}
\begin{itemize}
  \item Fongicide cuivrique (Bouillie bordelaise) : 5 g/L, 3 applications à 10j d'intervalle
  \item Éliminer et brûler les parties atteintes
  \item Améliorer le drainage (éviter stagnation eau)
\end{itemize}
\textbf{Prévention :} Taille d'entretien pour aérer la canopée, désherbage régulier\\[0.2cm]
\textit{Agents consultés : HealthAgent, CropAgent | Latence : 29.5s}
\end{successbox}

\subsection{Scénario 2 : Recommandation de Fertilisation}

\begin{warningbox}[Requête utilisateur]
\textit{« Quel engrais pour mon maïs de 1 ha au stade végétatif ? »}\\
Région : Ouest | Sol : Volcanique (pH 6.2)
\end{warningbox}

\begin{successbox}[Réponse du système (FertilizerAgent)]
\textbf{Besoins NPK calculés pour Maïs (1 ha, stade végétatif) :}\\[0.3cm]
\begin{tabular}{lrr}
\toprule
\textbf{Élément} & \textbf{Besoin/ha} & \textbf{Produit recommandé} \\
\midrule
Azote (N) & 120 kg/ha & Urée 46\% : 261 kg \\
Phosphore (P) & 60 kg/ha & TSP 46\% : 130 kg \\
Potassium (K) & 80 kg/ha & KCl 60\% : 133 kg \\
\bottomrule
\end{tabular}\\[0.3cm]
\textbf{Alternative organique :} Fumier de bovins (15 t/ha) + Cendres de bois (500 kg/ha)\\
\textbf{Calendrier :} Apport fractionné en 3 fois (semis, 30j, 60j)\\[0.2cm]
\textit{Agents consultés : FertilizerAgent | Latence : 19.4s}
\end{successbox}

\subsection{Scénario 3 : Prédiction de Culture Adaptée (ML)}

\begin{warningbox}[Données IoT de la parcelle]
pH : 5.8 | Humidité sol : 65\% | Température : 24°C | Précipitations : 1800mm/an\\
N : 45 mg/kg | P : 12 mg/kg | K : 90 mg/kg | Altitude : 750m | Région : Ouest
\end{warningbox}

\begin{successbox}[Prédiction ML (Random Forest)]
\textbf{Top-3 cultures recommandées :}
\begin{enumerate}
  \item \textbf{Café Arabica} — Probabilité : \textbf{87.3\%} ✅ (sol volcanique fertile, altitude idéale)
  \item Maïs — Probabilité : 8.1\% (possible mais sous-optimal)
  \item Haricot — Probabilité : 4.6\% (culture associée possible)
\end{enumerate}
\textbf{Justification :} Sol volcanique pH 5.8 optimal pour Arabica, altitude 750m idéale, pluviométrie 1800mm/an parfaite
\end{successbox}

\section{Évaluation de la Qualité des Réponses}

\subsection{Critères d'Évaluation}

Les réponses ont été évaluées par un panel de \textbf{5 agronomes camerounais} selon 4 critères :

\begin{table}[h!]
\centering
\caption{Évaluation de la qualité des réponses par des experts}
\begin{tabular}{lcccc}
\toprule
\rowcolor{AgriBlue!20}
\textbf{Agent} & \textbf{Pertinence} & \textbf{Précision} & \textbf{Concision} & \textbf{Praticité} \\
\midrule
WeatherAgent & 4.8/5 & 4.9/5 & 4.5/5 & 4.7/5 \\
\rowcolor{gray!10}
CropAgent & 4.6/5 & 4.5/5 & 4.4/5 & 4.6/5 \\
HealthAgent & 4.7/5 & 4.6/5 & 4.3/5 & 4.8/5 \\
\rowcolor{gray!10}
EconomicAgent & 4.5/5 & 4.7/5 & 4.6/5 & 4.5/5 \\
FertilizerAgent & 4.8/5 & 4.9/5 & 4.4/5 & 4.9/5 \\
\rowcolor{gray!10}
ResourcesAgent & 4.4/5 & 4.5/5 & 4.3/5 & 4.5/5 \\
\midrule
\textbf{Moyenne} & \textbf{4.63/5} & \textbf{4.68/5} & \textbf{4.42/5} & \textbf{4.67/5} \\
\bottomrule
\end{tabular}
\end{table}

\subsection{Satisfaction Utilisateurs}

\begin{figure}[h!]
\centering
\begin{tikzpicture}
\begin{axis}[
  xbar,
  width=10cm, height=6cm,
  xlabel={Score moyen (/5)},
  symbolic y coords={Praticité, Concision, Précision, Pertinence, Satisfaction globale},
  ytick=data,
  xmin=0, xmax=5,
  nodes near coords,
  nodes near coords align={horizontal},
  bar width=0.5cm,
  grid=major,
  grid style={dashed, gray!30},
]
\addplot[fill=AgriGreen!70, draw=AgriGreen] 
  coordinates {(4.67,Praticité) (4.42,Concision) (4.68,Précision) (4.63,Pertinence) (4.7,Satisfaction globale)};
\end{axis}
\end{tikzpicture}
\caption{Scores d'évaluation par critère (panel d'experts)}
\label{fig:evaluation_scores}
\end{figure}

\section{Limites Identifiées}

\begin{table}[h!]
\centering
\caption{Limites du système et pistes d'amélioration}
\begin{tabularx}{\textwidth}{lXX}
\toprule
\rowcolor{AgriBlue!20}
\textbf{Limite} & \textbf{Impact} & \textbf{Solution envisagée} \\
\midrule
Latence LLM (20--30s) & Expérience utilisateur & Cache étendu, modèle local (Ollama) \\
\rowcolor{gray!10}
Dépendance Internet & Zones rurales enclavées & Mode hors-ligne avec LLM embarqué \\
Langue unique (français) & Exclusion locuteurs Fulfulde/Pidgin & Traduction automatique intégrée \\
\rowcolor{gray!10}
Données ML limitées & Précision prédiction & Collecte terrain via IoT (en cours) \\
\bottomrule
\end{tabularx}
\end{table}
