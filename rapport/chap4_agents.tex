% ============================================================
%  CHAPITRE 4 — AGENTS SPÉCIALISÉS
% ============================================================
\chapter{Description des Agents Spécialisés}

\section{Vue d'Ensemble des Agents}

Le système comprend \textbf{6 agents spécialisés}, chacun expert dans un domaine précis de l'agronomie camerounaise. Tous héritent de la classe \texttt{BaseAgent} et partagent le même \texttt{LLMService}.

\begin{table}[h!]
\centering
\caption{Synthèse des 6 agents spécialisés}
\begin{tabularx}{\textwidth}{llXl}
\toprule
\rowcolor{AgriBlue!20}
\textbf{Agent} & \textbf{Domaine} & \textbf{Outils principaux} & \textbf{Intents} \\
\midrule
WeatherAgent & Météorologie & Open-Meteo API, prévisions 14j & 6 \\
\rowcolor{gray!10}
CropAgent & Agronomie & Calendriers, variétés, techniques & 5 \\
HealthAgent & Phytopathologie & Diagnostic, traitement, prévention & 5 \\
\rowcolor{gray!10}
EconomicAgent & Économie & Prix FCFA, rentabilité, marchés & 7 \\
ResourcesAgent & Pédologie & Sol, irrigation, amendements & 7 \\
\rowcolor{gray!10}
FertilizerAgent & Fertilisation & NPK, compost, carences & 6 \\
\bottomrule
\end{tabularx}
\end{table}

\section{WeatherAgent — Agent Météorologique}

\subsection{Rôle et Fonctionnement}
Le \texttt{WeatherAgent} est l'interface avec l'API météorologique Open-Meteo. Il fournit des données climatiques \textbf{en temps réel} pour chacune des 10 régions du Cameroun, avec des coordonnées GPS précises.

\begin{lstlisting}[language=Python, caption=Coordonnées GPS des régions dans WeatherAgent]
REGION_COORDINATES = {
    "Centre":      {"lat": 3.8480, "lon": 11.5021},
    "Littoral":    {"lat": 4.0511, "lon":  9.7679},
    "Ouest":       {"lat": 5.4777, "lon": 10.4176},
    "Nord-Ouest":  {"lat": 5.9631, "lon": 10.1591},
    "Adamaoua":    {"lat": 7.3167, "lon": 13.5833},
    "Nord":        {"lat": 9.3000, "lon": 13.4000},
    "Extreme-Nord":{"lat":10.5972, "lon": 14.3158},
    # ...
}
\end{lstlisting}

\subsection{Outils Disponibles}
\begin{itemize}
  \item \texttt{get\_agricultural\_weather\_summary} : Synthèse météo agricole (température, vent, pluie)
  \item \texttt{get\_weather\_forecast} : Prévisions à J+3, J+7, J+14
  \item \texttt{get\_irrigation\_advice} : Conseil d'irrigation basé sur l'évapotranspiration (ET0)
  \item \texttt{get\_climate\_alerts} : Détection de conditions dangereuses (vent $>$ 40 km/h, pluie $>$ 100mm/3j)
  \item \texttt{analyze\_rainfall\_patterns} : Analyse tendances pluviométriques sur 14 jours
  \item \texttt{get\_frost\_risk} : Risque de gel pour les régions montagneuses
\end{itemize}

\subsection{Calcul du Conseil d'Irrigation}
Le conseil d'irrigation est basé sur le bilan hydrique :
\begin{equation}
\text{Déficit} = ET_0 - P_{3j}
\end{equation}
où $ET_0$ est l'évapotranspiration de référence (mm/3j) et $P_{3j}$ les précipitations prévues sur 3 jours.

\begin{itemize}
  \item Si $P_{3j} < 0.5 \times ET_0$ : \textbf{Irrigation nécessaire}
  \item Si $0.5 \times ET_0 \leq P_{3j} < ET_0$ : \textbf{Irrigation modérée conseillée}
  \item Si $P_{3j} \geq ET_0$ : \textbf{Irrigation non nécessaire}
\end{itemize}

\section{CropAgent — Agent Cultures}

\subsection{Rôle}
Le \texttt{CropAgent} est l'agronome virtuel du système. Il conseille sur les itinéraires techniques, les calendriers de plantation, les variétés adaptées et les techniques culturales.

\subsection{Intents Reconnus}
\begin{description}
  \item[\texttt{CALENDAR}] Questions sur les dates de plantation, calendriers culturaux
  \item[\texttt{ROTATION}] Conseils sur la rotation des cultures et l'assolement
  \item[\texttt{VARIETY}] Recommandations de variétés adaptées à la région
  \item[\texttt{TECHNIQUE}] Itinéraires techniques (espacement, densité, entretien)
  \item[\texttt{GENERAL}] Questions générales sur les cultures
\end{description}

\subsection{Exemple de Réponse}
\begin{successbox}[Exemple : Calendrier Maïs — Région Centre]
\textbf{Maïs pluvial (Région Centre) :}\\
\faCalendar\ \textbf{Plantation :} Mars--Avril (grande saison) ou Août--Septembre (petite saison)\\
\faSeedling\ \textbf{Variétés :} CMS 8704, ATP, TZPB (résistantes à la striure)\\
\faRuler\ \textbf{Espacement :} 75 cm $\times$ 40 cm (densité : 33 000 plants/ha)\\
\faCalendarCheck\ \textbf{Récolte :} 90--110 jours après semis
\end{successbox}

\section{HealthAgent — Agent Santé des Plantes}

\subsection{Base de Connaissances Phytosanitaire}
L'agent dispose d'une base de connaissances exhaustive sur les maladies et ravageurs des cultures camerounaises :

\begin{table}[h!]
\centering
\caption{Principales maladies par culture (base de connaissances HealthAgent)}
\begin{tabular}{p{2.5cm}p{4cm}p{4cm}}
\toprule
\rowcolor{AgriBlue!20}
\textbf{Culture} & \textbf{Maladies principales} & \textbf{Ravageurs principaux} \\
\midrule
Cacao & Pourriture brune (\textit{Phytophthora}), Mirides, Balai de sorcière & \textit{Sahlbergella singularis} \\
\rowcolor{gray!10}
Café & Rouille orangée (\textit{Hemileia vastatrix}), Anthracnose & Scolytes (\textit{Hypothenemus hampei}) \\
Maïs & Charbon (\textit{Ustilago maydis}), Striure virale & Foreurs de tige (\textit{Sesamia calamistis}) \\
\rowcolor{gray!10}
Manioc & Mosaïque (CMV), Bactériose & Cochenilles (\textit{Phenacoccus manihoti}) \\
Plantain & Cercosporiose noire, Fusariose & Charançon (\textit{Cosmopolites sordidus}) \\
\bottomrule
\end{tabular}
\end{table}

\subsection{Processus de Diagnostic}
\begin{enumerate}
  \item \textbf{Extraction} : Identification de la culture et des symptômes (1 appel LLM combiné)
  \item \textbf{Diagnostic} : Identification du pathogène probable
  \item \textbf{Évaluation} : Niveau de gravité (Critique/Élevée/Modérée/Faible)
  \item \textbf{Traitement} : Produit + dose + méthode d'application
  \item \textbf{Prévention} : Mesures pour éviter les récidives
\end{enumerate}

\section{EconomicAgent — Agent Économique}

\subsection{Données de Marché}
L'agent économique travaille avec des données de marché en \textbf{FCFA} (Franc CFA), couvrant les marchés de Yaoundé, Douala, Bafoussam et Garoua.

\begin{table}[h!]
\centering
\caption{Prix de référence des cultures (FCFA/kg, marché Yaoundé)}
\begin{tabular}{lrrr}
\toprule
\rowcolor{AgriBlue!20}
\textbf{Culture} & \textbf{Prix min} & \textbf{Prix moyen} & \textbf{Prix max} \\
\midrule
Cacao (fèves sèches) & 1 500 & 1 800 & 2 200 \\
\rowcolor{gray!10}
Café Robusta & 1 200 & 1 500 & 1 800 \\
Maïs (grain sec) & 150 & 200 & 280 \\
\rowcolor{gray!10}
Manioc (frais) & 80 & 120 & 180 \\
Plantain (régime) & 1 000 & 1 500 & 2 500 \\
\rowcolor{gray!10}
Tomate & 300 & 500 & 900 \\
\bottomrule
\multicolumn{4}{r}{\small\textit{Source : Données de référence MINADER 2024}}
\end{tabular}
\end{table}

\section{FertilizerAgent — Agent Fertilisation}

\subsection{Calcul des Besoins NPK}
Le \texttt{FertilizerAgent} est le plus technique des agents. Il calcule les besoins précis en azote (N), phosphore (P) et potassium (K) pour chaque culture et superficie.

\begin{equation}
\text{Besoin total (kg)} = \text{Besoin/ha} \times \text{Superficie (ha)} \times \text{Facteur stade}
\end{equation}

\subsection{Outils Spécialisés}
\begin{itemize}
  \item \texttt{calculate\_npk\_requirements} : Calcul NPK selon culture, superficie et stade phénologique
  \item \texttt{get\_organic\_fertilizers} : Équivalents en engrais organiques locaux (fumier, compost)
  \item \texttt{diagnose\_nutrient\_deficiency} : Diagnostic de carence à partir des symptômes visuels
  \item \texttt{get\_application\_schedule} : Calendrier d'application fractionné
  \item \texttt{get\_soil\_amendment\_advice} : Conseils d'amendement selon type de sol
  \item \texttt{calculate\_compost\_recipe} : Recette de compost à partir des ressources locales
\end{itemize}

\section{ResourcesAgent — Agent Ressources et Sols}

\subsection{Gestion des Sols Camerounais}
Le \texttt{ResourcesAgent} est spécialisé dans la pédologie camerounaise. Il connaît les caractéristiques des principaux types de sols :

\begin{itemize}
  \item \textbf{Sols ferralitiques} (Centre, Sud, Est) : pH acide 4,5--5,5, riches en fer et aluminium
  \item \textbf{Sols volcaniques} (Ouest, Nord-Ouest) : Très fertiles, pH 5--7, bonne rétention d'eau
  \item \textbf{Sols ferrugineux} (Nord) : pH légèrement acide, carence en phosphore
  \item \textbf{Vertisols} (Extrême-Nord) : Argileux, pH alcalin 7--8,5, problèmes de salinité
\end{itemize}
