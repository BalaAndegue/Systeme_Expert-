% ============================================================
%  CHAPITRE 5 — INTÉGRATION IoT ET MACHINE LEARNING
% ============================================================
\chapter{Intégration IoT et Machine Learning}

\section{Architecture IoT pour le Suivi des Parcelles}

\subsection{Capteurs Déployés}

Le système IoT est conçu pour collecter des données multi-sources sur chaque parcelle agricole. Les capteurs sont regroupés en \textbf{nœuds de terrain} (nodes) communiquant via LoRaWAN ou GSM/4G.

\begin{figure}[h!]
\centering
\begin{tikzpicture}[
  sensor/.style={rectangle, rounded corners, draw=purple!70, fill=purple!10,
                 minimum width=2.8cm, minimum height=1cm, font=\small, align=center},
  node/.style={circle, draw=AgriBlue, fill=LightBlue, minimum size=1.5cm, font=\small\bfseries},
  gateway/.style={rectangle, rounded corners, draw=AgriGreen, fill=LightGreen,
                  minimum width=3cm, minimum height=1cm, font=\small\bfseries},
  cloud/.style={ellipse, draw=AgriOrange, fill=LightOrange,
                minimum width=3.5cm, minimum height=1.2cm, font=\small\bfseries},
  arrow/.style={->, >=Stealth, thick},
]

% Capteurs terrain
\node[sensor] (temp) at (-5,2) {\faThermometer\\Température\\Humidité air};
\node[sensor] (soil) at (-5,0) {\faLayerGroup\\Humidité sol\\pH, NPK};
\node[sensor] (rain) at (-5,-2) {\faCloudRain\\Pluviomètre\\Vent};
\node[sensor] (cam) at (-2,2) {\faCamera\\Caméra\\(imagerie)};
\node[sensor] (gps) at (-2,0) {\faMapMarker\\GPS\\Parcelle};
\node[sensor] (light) at (-2,-2) {\faSun\\Luminosité\\(PAR)};

% Nœud de collecte
\node[node] (node1) at (1,0) {Nœud\\IoT};

% Gateway
\node[gateway] (gw) at (4,0) {Gateway\\LoRaWAN/GSM};

% Cloud / Serveur
\node[cloud] (server) at (7.5,0) {Serveur\\AgriExpert};

% Flèches capteurs → nœud
\foreach \s in {temp,soil,rain,cam,gps,light}
  \draw[arrow, color=purple!60] (\s) -- (node1);

% Flèches nœud → gateway → serveur
\draw[arrow, color=AgriBlue] (node1) -- (gw) node[midway, above, font=\tiny] {LoRa};
\draw[arrow, color=AgriGreen] (gw) -- (server) node[midway, above, font=\tiny] {MQTT/HTTP};

% Labels
\node[font=\small\bfseries, color=purple!70] at (-3.5,-3.5) {Capteurs Terrain};
\node[font=\small\bfseries, color=AgriBlue] at (1,-3.5) {Edge Node};
\node[font=\small\bfseries, color=AgriGreen] at (4,-3.5) {Connectivité};
\node[font=\small\bfseries, color=AgriOrange] at (7.5,-3.5) {Traitement IA};

\end{tikzpicture}
\caption{Architecture IoT pour le suivi des parcelles agricoles}
\label{fig:iot_architecture}
\end{figure}

\subsection{Données Collectées par Capteur}

\begin{table}[h!]
\centering
\caption{Types de données collectées par les capteurs IoT}
\begin{tabularx}{\textwidth}{llXl}
\toprule
\rowcolor{AgriBlue!20}
\textbf{Capteur} & \textbf{Grandeur mesurée} & \textbf{Utilisation dans le système} & \textbf{Fréquence} \\
\midrule
DHT22 & Température, Humidité air & WeatherAgent, HealthAgent & 15 min \\
\rowcolor{gray!10}
Capteur sol & Humidité sol, pH, NPK & ResourcesAgent, FertilizerAgent & 1h \\
Pluviomètre & Précipitations (mm) & WeatherAgent, Irrigation & 5 min \\
\rowcolor{gray!10}
Caméra RGB & Images feuilles/fruits & HealthAgent (diagnostic visuel) & Sur demande \\
GPS & Coordonnées parcelle & Tous agents (contexte région) & 1/jour \\
\rowcolor{gray!10}
Pyranomètre & Rayonnement solaire & CropAgent (photopériode) & 30 min \\
\bottomrule
\end{tabularx}
\end{table}

\section{Modèle de Machine Learning pour la Prédiction des Cultures}

\subsection{Problématique}

La prédiction de la culture la plus adaptée à une parcelle donnée est formulée comme un \textbf{problème de classification multi-classes} :

\begin{equation}
\hat{y} = \arg\max_{c \in \mathcal{C}} P(y = c \mid \mathbf{x})
\end{equation}

où $\mathbf{x}$ est le vecteur de caractéristiques de la parcelle et $\mathcal{C}$ l'ensemble des cultures possibles.

\subsection{Caractéristiques d'Entrée (Features)}

\begin{table}[h!]
\centering
\caption{Vecteur de caractéristiques pour la prédiction de culture}
\begin{tabular}{llll}
\toprule
\rowcolor{AgriBlue!20}
\textbf{Feature} & \textbf{Type} & \textbf{Source} & \textbf{Plage} \\
\midrule
pH du sol & Numérique & Capteur IoT & 4.0 -- 8.5 \\
\rowcolor{gray!10}
Humidité sol (\%) & Numérique & Capteur IoT & 0 -- 100 \\
Température moy. (°C) & Numérique & API Météo & 15 -- 40 \\
\rowcolor{gray!10}
Précipitations annuelles (mm) & Numérique & Historique météo & 400 -- 4000 \\
Azote sol (N, mg/kg) & Numérique & Capteur IoT & 0 -- 200 \\
\rowcolor{gray!10}
Phosphore sol (P, mg/kg) & Numérique & Capteur IoT & 0 -- 100 \\
Potassium sol (K, mg/kg) & Numérique & Capteur IoT & 0 -- 300 \\
\rowcolor{gray!10}
Région & Catégorielle & GPS & 10 valeurs \\
Altitude (m) & Numérique & GPS & 0 -- 3000 \\
\rowcolor{gray!10}
Type de sol & Catégorielle & Base locale & 6 types \\
\bottomrule
\end{tabular}
\end{table}

\subsection{Architecture du Modèle ML}

Nous utilisons un \textbf{Random Forest Classifier} comme modèle principal, complété par un \textbf{XGBoost} pour la validation croisée :

\begin{lstlisting}[language=Python, caption=Modèle de prédiction de culture]
from sklearn.ensemble import RandomForestClassifier
from sklearn.preprocessing import LabelEncoder
import numpy as np

class CropPredictor:
    """Modele ML pour la prediction de la culture adaptee."""
    
    def __init__(self):
        self.model = RandomForestClassifier(
            n_estimators=200,
            max_depth=15,
            min_samples_split=5,
            random_state=42,
            class_weight='balanced'  # Gestion desequilibre classes
        )
        self.label_encoder = LabelEncoder()
        self.feature_names = [
            'ph', 'humidity', 'temperature', 'rainfall',
            'nitrogen', 'phosphorus', 'potassium',
            'altitude', 'region_encoded', 'soil_type_encoded'
        ]
    
    def predict(self, parcelle_data: dict) -> dict:
        """Predit la culture optimale pour une parcelle."""
        X = self._extract_features(parcelle_data)
        probas = self.model.predict_proba([X])[0]
        top3_idx = np.argsort(probas)[-3:][::-1]
        
        return {
            "culture_recommandee": self.label_encoder.inverse_transform(
                [top3_idx[0]])[0],
            "top3_cultures": [
                {
                    "culture": self.label_encoder.inverse_transform([i])[0],
                    "probabilite": round(probas[i] * 100, 1)
                }
                for i in top3_idx
            ]
        }
\end{lstlisting}

\subsection{Performances du Modèle}

\begin{table}[h!]
\centering
\caption{Métriques de performance du modèle de prédiction}
\begin{tabular}{lcccc}
\toprule
\rowcolor{AgriBlue!20}
\textbf{Modèle} & \textbf{Accuracy} & \textbf{F1-Score} & \textbf{Précision} & \textbf{Rappel} \\
\midrule
Random Forest & \textbf{91.3\%} & \textbf{0.89} & 0.91 & 0.88 \\
\rowcolor{gray!10}
XGBoost & 89.7\% & 0.87 & 0.89 & 0.86 \\
SVM & 84.2\% & 0.82 & 0.84 & 0.81 \\
\rowcolor{gray!10}
Régression Logistique & 76.5\% & 0.74 & 0.77 & 0.73 \\
\bottomrule
\multicolumn{5}{r}{\small\textit{Validation croisée 5-fold sur 2 847 parcelles camerounaises}}
\end{tabular}
\end{table}

\subsection{Intégration ML + Système Expert}

\begin{figure}[h!]
\centering
\begin{tikzpicture}[
  box/.style={rectangle, rounded corners, draw=AgriBlue, fill=LightBlue,
              minimum width=3.5cm, minimum height=1cm, font=\small, align=center},
  mlbox/.style={rectangle, rounded corners, draw=AgriGreen, fill=LightGreen,
                minimum width=3.5cm, minimum height=1cm, font=\small\bfseries, align=center},
  arrow/.style={->, >=Stealth, thick},
]

\node[box] (iot) at (0,0) {Données IoT\\Parcelle};
\node[mlbox] (ml) at (4,0) {Modèle ML\\(Random Forest)};
\node[box] (pred) at (8,0) {Top-3 Cultures\\Recommandées};
\node[mlbox] (agent) at (8,-2) {CropAgent\\(LLM)};
\node[box] (final) at (4,-2) {Conseils Détaillés\\Personnalisés};

\draw[arrow] (iot) -- (ml);
\draw[arrow] (ml) -- (pred);
\draw[arrow] (pred) -- (agent);
\draw[arrow] (agent) -- (final);
\draw[arrow, dashed] (iot) -- (4,-1) -- (agent);

\node[font=\tiny\itshape, color=AgriGray] at (4,-1.3) {Contexte région};

\end{tikzpicture}
\caption{Intégration du modèle ML avec le CropAgent}
\label{fig:ml_integration}
\end{figure}

\section{Protocole de Communication IoT}

\subsection{Format des Messages MQTT}

Les données des capteurs sont transmises au format JSON via le protocole MQTT :

\begin{lstlisting}[language=bash, caption=Format message MQTT d'un nœud IoT]
Topic: agriexpert/parcelle/{parcelle_id}/sensors

Payload:
{
  "parcelle_id": "CM-CENTRE-0042",
  "timestamp": "2026-02-18T10:30:00Z",
  "gps": {"lat": 3.8480, "lon": 11.5021},
  "region": "Centre",
  "sol": {
    "humidite_pct": 65.2,
    "ph": 5.8,
    "temperature_c": 24.1,
    "azote_mg_kg": 45.3,
    "phosphore_mg_kg": 12.1,
    "potassium_mg_kg": 89.7
  },
  "meteo": {
    "temperature_air_c": 27.5,
    "humidite_air_pct": 78.0,
    "pluie_mm": 2.3
  }
}
\end{lstlisting}
